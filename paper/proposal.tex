\documentclass[11pt,a4paper]{article}
\usepackage[utf8x]{inputenc}
\usepackage{ucs}
\usepackage{amsmath}
\usepackage{amsfonts}
\usepackage{amssymb}
\usepackage{url}

\newcommand{\question}{\textbf{---NEED-REVIEW-HERE---}}

\author{HU, Pili}
\title{Spectral Techniques for Community Detection on 2-Hop Topology}

\begin{document}

\maketitle

\begin{abstract}
	In one of our former project\cite{hu2011-cd2hop}, we formulated
	the problem of community detection on 2-hop topology. Given 
	one observer in a social network, and the graph it can reach 
	within two steps, we want to determine for those vertices whether they're
	in the same community with the observer. 
	The previous project extracts features like Common Neighbours, Ademic/Adar
	Score, Pagerank, and Personalized Pagerank, etc. Then several neural 
	networks are trained to combine those features. 
	Interested readers can refer to \cite{hu2011-cd2hop}
	for more information.  
	
	In this project, we augment our research based on the same settings. 
	Several spectral techniques are leveraged to tackle with the 2-hop 
	problem. For instance, spectral clustering may utilize the information 
	hidden in more eigen vectors. Other technique like personalized PageRank 
	can be studied with the budgeted learning context. 
	For data, codes, and other materials, please refer to 
	our open source repository\cite{hu2012-spectral2hop}. 
\end{abstract}

\pagebreak
\tableofcontents
\pagebreak

\section{Introduction}

\subsection{Problem}
Notations:
\begin{table}[htb]
	\centering
	\caption{Notations}
	\label{tbl:notation}
	\begin{tabular}{c|c}
	\hline
	Symbol & Explanation \\
	\hline
	$n$ & node number \\
	\hline
	$m$ & edge number \\
	\hline
	$A$ & adjacency matrix \\
	\hline
	$A_{ij}$ & 1 if node $i$ and node $j$ are directly \\
	& connected, 0 otherwise\\
	\hline
	$N(i)$ & neighbourhood of $i$, namely \\
	& $N(i) = \{j | A_{ij} = 1 \}$ \\
	\hline
	$d(i)$ & degree of node $i$ \\
	& $d(i) = \sum_{j}{A_{ij}}$ \\
	\hline
	$o$ & observer \\
	\hline
	$l_i$ & the real label of node $i$ \\
	\hline
	$L_i$ & the predicted label of node $i$ \\
	\hline
	$N2(i)$ & the 2-hop neighbours we study, defined as \\
	& $\cup_{j \in N(i) \cup \{i\}}{N(j)}$\\
	\hline
	\end{tabular}
\end{table}

Problem settings:
\begin{itemize}
	\item Give observer $o$. 
	\item Give the two hop topology, namely the $N2(o)$ and 
	the associated links. 
	\item Try to predict $L_i$ according to the above information. 
	\item Ground truth $l_i$ is directly crawled from the website. 
\end{itemize}

\subsection{Evaluation}

Our evaluation method distinguishes this work from many 
previous ones. Usual evaluation criterion is defining 
a certain quality functions, like\cite{aggarwal2011social} normalized cut, 
conductance, and modularity. Different research groups 
have their own taste and preference. 

The major problem is, those quality functions can reflect 
the accuracy of algorithms only to a limited extent. 
Let spectral clustering be an example. Luxburg\cite{von2007tutorial}
explains clustering using the notion of finding sets of nodes 
with small conductance in between. In this way, if conductance 
is selectd as the quality function, a good "clustering" is 
a natural consequence of spectral clustering algorithm. 

While those quality functions are simple and intuitive to 
facilitate theoretical study, they can not completely reflect the 
real application performance. In this work, we crawled ground truth
data from one SNS, thus we're able to evaluate the output 
against those crawled label. 



\subsection{Digest of Previous Results}

\section{Proposed Future Study Line}



\input{../reference/gen_bib.bbl}

\end{document}
